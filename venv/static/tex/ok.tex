\documentclass{article}
\usepackage[utf8]{inputenc}
\usepackage[T1]{fontenc}
\usepackage{amsmath}
\usepackage{amssymb}

\usepackage[utf8]{inputenc}
\usepackage[T1]{fontenc}
\begin{document}

\title{Examen de Mathématiques – Niveau 1ère (Évaluation 1)}
\date{}
\maketitle

\section*{Exercice 1 : Fonction et Dérivée}
Soit \( f(x) = x^3 - 3x^2 + 2x \).

\textbf{Questions :}
\begin{enumerate}
    \item[a)] Calculer \( f'(x) \).
    \item[b)] Étudier les variations de \( f \).
    \item[c)] Résoudre \( f(x) = 0 \).
\end{enumerate}

\textbf{Réponses de l’élève :}
\begin{enumerate}
    \item[a)] \( f'(x) = 3x^2 - 6x + 2 \).
    \item[b)] \( f \) est décroissante sur \( [0,1] \) et croissante sur \( [1,+\infty[ \) (approximation).
    \item[c)] \( f(x) = x(x-1)(x-2) = 0 \) donc \( x = 0, 1, 2 \).
\end{enumerate}

\section*{Exercice 2 : Équation du Second Degré}
Résoudre \( x^2 - 5x + 6 = 0 \).

\textbf{Réponse de l’élève :}
\( x = 2 \) ou \( x = 3 \).

\section*{Exercice 3 : Géométrie Analytique}
Soit le triangle \( ABC \) avec \( A(0,0) \), \( B(4,0) \) et \( C(2,3) \).

\textbf{Questions :}
\begin{enumerate}
    \item[a)] Montrer que \( ABC \) est isocèle.
    \item[b)] Calculer son aire.
    \item[c)] Donner les coordonnées du milieu de \( [BC] \).
\end{enumerate}

\textbf{Réponses de l’élève :}
\begin{enumerate}
    \item[a)] \( AB = 4 \) et \( AC \approx \sqrt{13} \); \( BC \approx \sqrt{13} \) donc isocèle.
    \item[b)] Aire \( \approx 6 \).
    \item[c)] Milieu de \( BC = (3, 1.5) \).
\end{enumerate}

\section*{Exercice 4 : Probabilités}
Dans un sac, il y a 4 boules rouges et 6 boules bleues.

\textbf{Questions :}
\begin{enumerate}
    \item[a)] Calculer \( P(\text{rouge}) \).
    \item[b)] Avec remise, calculer \( P(2 \text{ rouges}) \).
\end{enumerate}

\textbf{Réponses de l’élève :}
\begin{enumerate}
    \item[a)] \( P(\text{rouge}) = \frac{4}{10} = 0.4 \).
    \item[b)] \( 0.4 \times 0.4 = 0.16 \).
\end{enumerate}

\end{document}