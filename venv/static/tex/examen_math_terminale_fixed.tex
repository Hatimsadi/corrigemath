\documentclass{article}
\usepackage[utf8]{inputenc}
\usepackage[T1]{fontenc}
\usepackage{amsmath}
\usepackage{amssymb}

\usepackage[utf8]{inputenc}
\usepackage[T1]{fontenc}
\begin{document}

\section*{Examen de Mathématiques - Niveau Terminale}

\subsection*{Exercice 1 : Dérivation}
Soit $f(x) = x^3 - 3x^2 + 2x - 5$.\\
1) Calculer $f'(x)$.\\
Réponse : $f'(x) = 3x^2 - 6x + 2$.\\
2) Résoudre $f'(x) = 0$.\\
Réponse : $3x^2 - 6x + 2 = 0 \rightarrow \Delta = 36 - 24 = 12$\\
   $x = \frac{6 \pm \sqrt{12}}{6} \rightarrow x \approx 1.577 \text{ et } x \approx 0.423$

\subsection*{Exercice 2 : Probabilités}
Une urne contient 5 boules rouges, 3 bleues et 2 vertes.\\
1) Quelle est la probabilité de tirer une boule rouge ?\\
Réponse : $P(R) = \frac{5}{10} = 0.5$.\\
2) Quelle est la probabilité de tirer une bleue puis une verte (sans remise) ?\\
Réponse : $P(B \text{ et } V) = \left(\frac{3}{10}\right) \cdot \left(\frac{2}{9}\right) = \frac{6}{90} = 0.0667$.

\subsection*{Exercice 3 : Géométrie}
Un triangle a pour sommets $A(1,2)$, $B(4,6)$, $C(7,2)$.\\
1) Calculer la longueur $AB$.\\
Réponse : $AB = \sqrt{(4-1)^2 + (6-2)^2} = \sqrt{9 + 16} = \sqrt{25} = 5$.\\
2) Montrer que $ABC$ est isocèle.\\
Réponse : $AC = \sqrt{(7-1)^2 + (2-2)^2} = \sqrt{36} = 6$.\\
   $BC = \sqrt{(7-4)^2 + (2-6)^2} = \sqrt{9 + 16} = \sqrt{25} = 5$.\\
   Donc $AB = BC$, $ABC$ est isocèle.

\subsection*{Exercice 4 : Limites et Suites}
On définit $u_n = \frac{3n + 1}{2n - 1}$.\\
1) Calculer la limite de $u_n$.\\
Réponse : $\lim_{n \to \infty} \frac{3n + 1}{2n - 1} = \frac{3}{2}$ (en divisant haut et bas par $n$).

Fin de l'examen.

\end{document}