\documentclass{article}
\usepackage[utf8]{inputenc}
\usepackage[T1]{fontenc}
\usepackage{amsmath}
\usepackage{amssymb}
\usepackage{xcolor}

\usepackage[utf8]{inputenc}
\usepackage[T1]{fontenc}
\begin{document}

\title{Examen de Mathématiques – Niveau 1ère (Évaluation 1)}
\maketitle

\section*{Exercice 1 : Fonction et Dérivée}

Soit $f(x) = x^3 - 3x^2 + 2x$. 

\subsection*{Questions :}
\begin{enumerate}
    \item a) Calculer $f'(x)$.
    \item b) Étudier les variations de $f$.
    \item c) Résoudre $f(x) = 0$.
\end{enumerate}

\subsection*{Réponses de l’élève :}
\begin{enumerate}
    \item a) $f'(x) = 3x^2 - 6x + 2$. \\
    \textcolor{red}{Réponse correcte. Note : 4/4}
    
    \item b) $f$ est décroissante sur $[0, 1]$ et croissante sur $[1, +\infty[$ (approximation). \\
    \textcolor{red}{Il manque une justification rigoureuse des intervalles de variation. Il faut étudier le signe de $f'(x)$. \\
    $f'(x) = 3x^2 - 6x + 2 \Rightarrow \Delta = 36 - 4 \times 3 \times 2 = 12$\\
    Les racines sont $x = \frac{6 \pm \sqrt{12}}{6} = 1 \pm \frac{\sqrt{3}}{3}$. \\
    $f$ est décroissante sur $\left] -\infty, 1 - \frac{\sqrt{3}}{3} \right]$ et croissante sur $\left[ 1 - \frac{\sqrt{3}}{3}, +\infty \right[$. \\
    Note : 2/4}
    
    \item c) $f(x) = x(x - 1)(x - 2) = 0$ donc $x = 0, 1, 2$. \\
    \textcolor{red}{Réponse correcte. Note : 4/4}
\end{enumerate}

\section*{Exercice 2 : Équation du Second Degré}

Résoudre $x^2 - 5x + 6 = 0$. 

\subsection*{Réponse de l’élève :}
$x = 2$ ou $x = 3$.\\
\textcolor{red}{Réponse correcte. Note : 4/4}

\section*{Exercice 3 : Géométrie Analytique}

Soit le triangle ABC avec $A(0,0)$, $B(4,0)$ et $C(2,3)$. 

\subsection*{Questions :}
\begin{enumerate}
    \item a) Montrer que ABC est isocèle.
    \item b) Calculer son aire.
    \item c) Donner les coordonnées du milieu de $[BC]$.
\end{enumerate}

\subsection*{Réponses de l’élève :}
\begin{enumerate}
    \item a) $AB = 4$ et $AC \approx \sqrt{13}$; $BC \approx \sqrt{13}$ donc isocèle. \\
    \textcolor{red}{Les longueurs $AC$ et $BC$ sont correctes. Réponse correcte. Note : 4/4}
    
    \item b) Aire $\approx 6$. \\
    \textcolor{red}{L'aire correcte est $6$. \\
    À l'aide de la formule : $\text{Aire} = \frac{1}{2} \left| 0(0-3) + 4(3-0) + 2(0-0) \right| = 6$. \\
    Note : 4/4}
    
    \item c) Milieu de $BC = (3, 1.5)$. \\
    \textcolor{red}{Réponse correcte. Note : 4/4}
\end{enumerate}

\section*{Exercice 4 : Probabilités}

Dans un sac, il y a 4 boules rouges et 6 boules bleues. 

\subsection*{Questions :}
\begin{enumerate}
    \item a) Calculer $P(\text{rouge})$.
    \item b) Avec remise, calculer $P(2 \text{ rouges})$.
\end{enumerate}

\subsection*{Réponses de l’élève :}
\begin{enumerate}
    \item a) $P(\text{rouge}) = \frac{4}{10} = 0.4$. \\
    \textcolor{red}{Réponse correcte. Note : 2/2}
    
    \item b) $0.4 \times 0.4 = 0.16$. \\
    \textcolor{red}{Réponse correcte. Note : 2/2}
\end{enumerate}

\section*{Note Finale}

\textcolor{red}{Total des points : 36/40 \\
Note : 18/20}

\end{document}