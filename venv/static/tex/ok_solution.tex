\documentclass{article}
\usepackage[utf8]{inputenc}
\usepackage[T1]{fontenc}
\usepackage{amsmath}
\usepackage{amssymb}

\usepackage[utf8]{inputenc}
\usepackage[T1]{fontenc}
\usepackage[utf8]{inputenc}
\usepackage[T1]{fontenc}
\begin{document}

\title{Examen de Mathématiques – Niveau 1ère (Évaluation 1)}
\maketitle

\section*{Exercice 1 : Fonction et Dérivée}
Soit $f(x) = x^3 - 3x^2 + 2x$. \\
Questions :
\begin{enumerate}
    \item[a)] Calculer $f'(x)$.
    \item[b)] Étudier les variations de $f$.
    \item[c)] Résoudre $f(x) = 0$.
\end{enumerate}
Réponses de l’élève :
\begin{enumerate}
    \item[a)] $f'(x) = 3x^2 - 6x + 2$. \\
    \textbf{Correction :} Réponse correcte.\\
    \textbf{Note : 1/1}
    
    \item[b)] $f$ est décroissante sur $[0,1]$ et croissante sur $[1,+\infty[$ (approximation). \\
    \textbf{Correction :} Calculons les variations de $f$ en déterminant le signe de $f'(x) = 3x^2 - 6x + 2$. Résolvons $f'(x) = 0$ :
    \[
    3x^2 - 6x + 2 = 0 \quad \Rightarrow \quad x = \frac{6 \pm \sqrt{(-6)^2 - 4 \times 3 \times 2}}{2 \times 3} = \frac{6 \pm \sqrt{12}}{6} = \frac{6 \pm 2\sqrt{3}}{6} = 1 \pm \frac{\sqrt{3}}{3}.
    \]
    $f'(x)$ change de signe à ces points : $1 - \frac{\sqrt{3}}{3}$ et $1 + \frac{\sqrt{3}}{3}$. Par analyse du signe de $f'(x)$, $f$ est croissante sur $]-\infty, 1 - \frac{\sqrt{3}}{3}] \cup [1 + \frac{\sqrt{3}}{3}, +\infty[$ et décroissante sur $[1 - \frac{\sqrt{3}}{3}, 1 + \frac{\sqrt{3}}{3}]$.\\
    \textbf{Correction nécessaire : la justification et les points exacts de changement de variations manquent.} \\
    \textbf{Note : 0.5/1}
    
    \item[c)] $f(x) = x(x-1)(x-2) = 0$ donc $x = 0, 1, 2$. \\
    \textbf{Correction :} Réponse correcte.\\
    \textbf{Note : 1/1}
\end{enumerate}

\section*{Exercice 2 : Équation du Second Degré}
Résoudre $x^2 - 5x + 6 = 0$. \\ 
Réponse de l’élève : 
$x = 2$ ou $x = 3$. \\
\textbf{Correction :} Réponse correcte. Les racines peuvent être trouvées par factorisation eu égard à $x^2 - 5x + 6 = (x - 2)(x - 3) = 0$ donc $x = 2$ ou $x = 3$. \\
\textbf{Note : 1/1}

\section*{Exercice 3 : Géométrie Analytique}
Soit le triangle ABC avec $A(0,0)$, $B(4,0)$ et $C(2,3)$. \\
Questions :
\begin{enumerate}
    \item[a)] Montrer que ABC est isocèle.
    \item[b)] Calculer son aire.
    \item[c)] Donner les coordonnées du milieu de $[BC]$.
\end{enumerate}
Réponses de l’élève :
\begin{enumerate}
    \item[a)] $AB = 4$ et $AC \approx \sqrt{13}$ ; $BC \approx \sqrt{13}$ donc isocèle. \\
    \textbf{Correction :} $AB = 4$, $AC = \sqrt{(2-0)^2 + (3-0)^2} = \sqrt{13}$ et $BC = \sqrt{(2-4)^2 + (3-0)^2} = \sqrt{13}$. ABC est isocèle car $AC = BC$. \\
    \textbf{Correction nécessaire : la justification est correcte mais il manquait des calculs complets.}\\
    \textbf{Note : 1/1}
    
    \item[b)] Aire $\approx 6$. \\
    \textbf{Correction :} L'aire du triangle est donnée par $\text{Aire} = \frac{1}{2} \left| x_1(y_2-y_3) + x_2(y_3-y_1) + x_3(y_1-y_2) \right|$.\\
    Appliquons la formule :
    \[
    \text{Aire} = \frac{1}{2} \left| 0(0-3) + 4(3-0) + 2(0-0) \right| = \frac{1}{2} \left| 0 + 12 + 0 \right| = \frac{1}{2} \times 12 = 6.
    \]
    Réponse correcte. \\
    \textbf{Note : 1/1}
    
    \item[c)] Milieu de $BC = (3, 1.5)$. \\
    \textbf{Correction :} Le milieu de $BC$ est $\left(\frac{2+4}{2}, \frac{3+0}{2}\right) = (3, 1.5)$.\\
    Réponse correcte. \\
    \textbf{Note : 1/1}
\end{enumerate}

\section*{Exercice 4 : Probabilités}
Dans un sac, il y a 4 boules rouges et 6 boules bleues. \\
Questions :
\begin{enumerate}
    \item[a)] Calculer $P(\text{rouge})$.
    \item[b)] Avec remise, calculer $P(2 \text{ rouges})$.
\end{enumerate}
Réponses de l’élève :
\begin{enumerate}
    \item[a)] $P(\text{rouge}) = \frac{4}{10} = 0.3$. \\
    \textbf{Correction :} $P(\text{rouge}) = \frac{4}{10} = 0.4$. Corrigez l'erreur de calcul.\\
    \textbf{Note : 0.5/1}
    
    \item[b)] $0.4 \times 0.4 = 0.2$. \\
    \textbf{Correction :} En supposant une remise, $P(2 \text{ rouges}) = 0.4 \times 0.4 = 0.16$. Veuillez vérifier votre multiplication. \\
    \textbf{Note : 0/1}
\end{enumerate}

\section*{Note Finale}
\begin{itemize}
    \item Exercice 1 : 2.5/3
    \item Exercice 2 : 1/1
    \item Exercice 3 : 3/3
    \item Exercice 4 : 0.5/2
\end{itemize}
\textbf{Note Totale : 7/9}
\end{document}