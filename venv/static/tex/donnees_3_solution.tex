\documentclass{article}
\usepackage[utf8]{inputenc}
\usepackage[T1]{fontenc}
\usepackage{amsmath}
\usepackage{amssymb}

\usepackage[utf8]{inputenc}
\usepackage[T1]{fontenc}
\begin{document}

Examen de Mathématiques – Terminale (Évaluation 2)

Exercice 1 : Dérivées et Tangentes

Soit \( f(x) = x^2 \ln(x) \) pour \( x > 0 \).

Questions :
\begin{enumerate}
    \item Calculer \( f'(x) \).
    \item Donner l’équation de la tangente en \( x = 1 \).
\end{enumerate}

Réponses de l’élève :
\begin{enumerate}
    \item \( f'(x) = 2x \ln(x) + x \).
    \item Pour \( x = 1, f(1) = 0 \) et \( f'(1) = 1 \) ; tangente \( y = x - 1 \).
\end{enumerate}

Correction:
\begin{enumerate}
    \item La dérivée \( f'(x) \) est obtenue en utilisant la règle du produit: 
    \[
    f'(x) = (x^2)' \ln(x) + x^2 (\ln(x))' = 2x \ln(x) + x^2 \frac{1}{x} = 2x \ln(x) + x.
    \]
    La réponse de l'élève est correcte. \textbf{Note : 1/1}
    
    \item Pour \( x = 1 \), calculons :
    \[
    f(1) = 1^2 \cdot \ln(1) = 0,
    \]
    \[
    f'(1) = 2 \cdot 1 \cdot \ln(1) + 1 = 0 + 1 = 1.
    \]
    L'équation de la tangente est donnée par la formule : \( y = f'(1)(x - 1) + f(1) \). Ainsi la tangente est :
    \[
    y = 1 \cdot (x - 1) + 0 = x - 1.
    \]
    La réponse de l'élève est correcte. \textbf{Note : 1/1}
\end{enumerate}

Exercice 2 : Intégrales

Calculer \( I = \int_{0}^{1} (3x^2 + 2x + 1) \, dx \).

Réponse de l’élève :
\[
I = \left[ x^3 + x^2 + x \right]_{0}^{1} = 1 + 1 + 1 = 3.
\]

Correction:
La primitive de la fonction \( 3x^2 + 2x + 1 \) est :
\[
\int (3x^2 + 2x + 1) \, dx = x^3 + x^2 + x + C.
\]
En évaluant l'intégrale définie de 0 à 1, nous avons :
\[
I = \left[ x^3 + x^2 + x \right]_{0}^{1} = \left(1^3 + 1^2 + 1\right) - \left(0^3 + 0^2 + 0\right) = 1 + 1 + 1 - 0 = 3.
\]
La réponse de l'élève est correcte. \textbf{Note : 1/1}

\bigskip

\textbf{Note totale : 3/3}

\end{document}