\documentclass{article}
\usepackage[utf8]{inputenc}
\usepackage[T1]{fontenc}
\usepackage{amsmath}
\usepackage{amssymb}

\usepackage[utf8]{inputenc}
\usepackage[T1]{fontenc}
\begin{document}

\title{Devoir de Mathématiques - Terminale}
\date{Durée : 2 heures}
\maketitle

\section*{Exercice 1 : Analyse (5 points)}
Soit la fonction définie sur $\mathbb{R}$ par : 
\[
f(x) = x^3 - 3x^2 + 2
\]

\subsection*{1. Calculer la dérivée $f'(x)$.}
Réponse : $f'(x) = 3x^2 - 6x$

\subsection*{2. Déterminer les coordonnées du point d'intersection avec l'axe des ordonnées.}
Réponse : Quand $x = 0$, $f(0) = 2$. Donc le point est $(0; 2)$.

\section*{Exercice 2 : Probabilités (7 points)}
Une urne contient 5 boules rouges et 3 boules bleues. On tire successivement 2 boules avec remise.

\subsection*{1. Calculer la probabilité d'obtenir deux boules de la même couleur.}
Réponse : 
\[
P = \binom{5}{2}\left(\frac{5}{8}\right)^2 + \binom{3}{2}\left(\frac{3}{8}\right)^2 = \frac{34}{64}
\]

\subsection*{2. Quelle est la probabilité d'obtenir au moins une boule rouge ?}
Réponse : 
\[
1 - \left(\frac{3}{8}\right)^2 = \frac{55}{64}
\]

\section*{Exercice 3 : Géométrie (8 points)}
Soit les points $A(1; 2)$, $B(4; 5)$ et $C(2; 7)$.

\subsection*{1. Calculer les coordonnées du vecteur $-\vec{AB}$.}
Réponse : 
\[
-\vec{AB} = (4 - 1; 5 - 2) = (3; 3)
\]

\subsection*{2. Montrer que $ABC$ est un triangle rectangle.}
Réponse : 
\[
AB = \sqrt{(3)^2 + (3)^2} = \sqrt{18}
\]
\[
BC = \sqrt{(-2)^2 + (2)^2} = \sqrt{8}
\]
\[
AC = \sqrt{(1)^2 + (5)^2} = \sqrt{26}
\]
On vérifie : $AB^2 + BC^2 = 18 + 8 = 26 = AC^2$.

\end{document}