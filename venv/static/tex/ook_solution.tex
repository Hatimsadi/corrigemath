\documentclass{article}
\usepackage[utf8]{inputenc}
\usepackage[T1]{fontenc}
\usepackage{amssymb}
\usepackage{xcolor}
\usepackage{amsmath}

\usepackage[utf8]{inputenc}
\usepackage[T1]{fontenc}
\begin{document}

\title{Correction Examen de Mathématiques – Niveau 1ère (Évaluation 2)}
\maketitle

\section*{Exercice 1 : Fonctions}

\begin{enumerate}
    \item $g'(x) = 4x - 4.$ 
    
    \textbf{Réponse de l'élève :} correcte. 
    
    \textbf{Note : 1.5/1.5}

    \item L'élève répond : décroissante sur $]-\infty,1]$, croissante sur $[1,+\infty[.$
    
    La dérivée est égale à 0 en $x=1$, elle est négative sur $]-\infty,1[$ et positive sur $]1,+\infty[$. \\ Ainsi, $g$ est décroissante sur $]-\infty,1[$ \textcolor{red}{\textbf{(Attention, vous avez inclus le point 1 incorrectement)}} et croissante sur $]1,+\infty[$ \textcolor{red}{\textbf{(Vous avez aussi incorrectement inclus le point 1)}}.
    
    \textcolor{red}{\textbf{Correction : } Décroissante sur $]-\infty,1[$ et croissante sur $]1,+\infty[$.}
    
    \textbf{Note : 1/2}

    \item L'élève donne une approximation sans justification. 
    
    \textcolor{red}{\textbf{Correction : }Résolvons exactement : 
    \[
    2x^2 -4x+1=0 \Longleftrightarrow \Delta = (-4)^2 - 4\times 2\times 1 = 16 -8 =8>0.
    \]
    On a donc deux solutions réelles : 
    \[
    x_1=\frac{4-\sqrt{8}}{4}=\frac{4-2\sqrt{2}}{4}=1-\frac{\sqrt{2}}{2},\quad x_2=\frac{4+2\sqrt{2}}{4}=1+\frac{\sqrt{2}}{2}
    \]}

    \textcolor{red}{\textbf{Attention : Toujours utiliser une résolution exacte en priorité.}}

    \textbf{Note : 0/1.5}
\end{enumerate}

\textbf{Total Exercice 1 :} 2.5/5

\section*{Exercice 2 : Équation}

Réponse correcte avec justification minimale.
\[
x^2 - 4 = 0 \Longleftrightarrow x^2 = 4 \Longleftrightarrow x = \pm 2
\]

\textcolor{red}{\textbf{Bien, justification correcte mais minimale.}}

\textbf{Note : 2/2}

\textbf{Total Exercice 2 :} 2/2

\section*{Exercice 3 : Géométrie}

\begin{enumerate}
    \item L'élève répond : Côtés opposés parallèles et égaux, c'est un rectangle.
    
    Ce n'est pas suffisant comme preuve.\textcolor{red}{\textbf{ Ajoutez: Il faut prouver au préalable que les vecteurs $\overrightarrow{AB}$ et $\overrightarrow{AD}$ sont orthogonaux en montrant que leur produit scalaire est nul :}}
    
    \textcolor{red}{
    \[
    \overrightarrow{AB}=(4,0);\quad \overrightarrow{AD}=(0,3)\quad  \overrightarrow{AB}\cdot \overrightarrow{AD}=4\times 0+0\times 3=0
    \]}
    
    \textcolor{red}{ Donc les côtés sont orthogonaux, ce qui prouve définitivement qu'on a un rectangle.}
    
    \textbf{Note : 1/2}
    
    \item L'aire est correcte.
    
    \textbf{Note : 1.5/1.5}
\end{enumerate}

\textbf{Total Exercice 3 :} 2.5/3.5

\section*{Exercice 4 : Probabilités}

\begin{enumerate}
    \item $P(\text{verte})=0.5$ correct.
    
    \textbf{Note : 1.5/1.5}
    
    \item La réponse donnée est bonne mais non justifiée.
    
    \textcolor{red}{\textbf{Correction : }}\textcolor{red}{
    $P(\text{2 vertes})=P(\text{verte})\times P(\text{verte})=0.5\times 0.5=0.25,$ puisque les tirages sont indépendants (remise de la boule).}

    \textcolor{red}{\textbf{Attention : Toujours justifier explicitement vos calculs.}}

    \textbf{Note : 0.75/1}
\end{enumerate}

\textbf{Total Exercice 4 :} 2.25/2.5

\section*{Récapitulatif des notes :}

\begin{itemize}
\item Exercice 1 : 2.5/5
\item Exercice 2 : 2/2
\item Exercice 3 : 2.5/3.5
\item Exercice 4 : 2.25/2.5
\end{itemize}

\textbf{\large Note finale : } 9.25/13 soit environ \textbf{14.2/20}.

\textcolor{red}{
\section*{Recommandations pour l'élève :}
\begin{itemize}
    \item Soyez plus rigoureux dans vos justifications, surtout en géométrie et en probabilités.
    \item N'oubliez pas d'effectuer des résolutions exactes en priorité plutôt que des approximations.
    \item Faites attention aux intervalles ouverts ou fermés concernant les intervalles de croissance et de décroissance.
    \item Relisez-vous soigneusement afin d'éviter des points perdus sur des erreurs d'inattention.
\end{itemize}
}

\end{document}