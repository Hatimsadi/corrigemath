\documentclass{article}
\usepackage[utf8]{inputenc}
\usepackage[T1]{fontenc}
\usepackage{amsmath}
\usepackage{amssymb}
\usepackage{xcolor}

\usepackage[utf8]{inputenc}
\usepackage[T1]{fontenc}
\begin{document}

\section*{Examen de Mathématiques – Niveau 1ère (Évaluation 2)}

\subsection*{Exercice 1 : Fonctions}
Soit $g(x) = 2x^2 - 4x + 1$. \\
Questions : 
\begin{enumerate}
    \item[a)] Calculer $g'(x)$.
    \item[b)] Donner les intervalles de croissance et décroissance.
    \item[c)] Résoudre $g(x) = 0$.
\end{enumerate}

Réponses de l’élève :
\begin{enumerate}
    \item[a)] $g'(x) = 4x - 4$. \\
    \textcolor{red}{Correct, note : 2/2}
    
    \item[b)] $g$ est décroissante sur $]-\infty, 1]$ et croissante sur $[1, +\infty[$.\\
    \textcolor{red}{Il fallait montrer la dérivée est positive ou négative sur chaque intervalle pour justifier.}\\
    \textcolor{red}{Note : 1/2}
    
    \item[c)] $g(x) = 0$ approx. pour $x = 0.5$ et $x = 1.5$.\\
    \textcolor{red}{Correction: $g(x) = 0 \Rightarrow 2x^2 - 4x + 1 = 0$.}
    \textcolor{red}{Utiliser le discriminant $\Delta = b^2 - 4ac = (-4)^2 - 4 \times 2 \times 1 = 8$,}
    \textcolor{red}{donc $x = \frac{4 \pm \sqrt{8}}{4} = \frac{4 \pm 2\sqrt{2}}{4} = 1 \pm \frac{\sqrt{2}}{2}$.}\\
    \textcolor{red}{Note : 0/2}
\end{enumerate}

\subsection*{Exercice 2 : Équation}
Résoudre $x^2 - 4 = 0$. \\
Réponse de l’élève : $x = \pm 2$. \\
\textcolor{red}{Correct, note : 2/2}

\subsection*{Exercice 3 : Géométrie}
On considère le quadrilatère $ABCD$ avec $A(0,0)$, $B(4,0)$, $C(4,3)$ et $D(0,3)$. \\
Questions :
\begin{enumerate}
    \item[a)] Vérifier que $ABCD$ est un rectangle.
    \item[b)] Calculer son aire.
\end{enumerate}

Réponses de l’élève :
\begin{enumerate}
    \item[a)] Les côtés opposés sont parallèles et égaux, c’est un rectangle.\\
    \textcolor{red}{Il faudrait montrer que tous les angles sont droits ou utiliser le produit scalaire pour justifier.} \\
    \textcolor{red}{Note : 1/2}

    \item[b)] Aire = $4 \times 3 = 12$. \\
    \textcolor{red}{Correct, note : 2/2}
\end{enumerate}

\subsection*{Exercice 4 : Probabilités}
Dans un sac, 5 boules vertes et 5 boules jaunes. \\
Questions :
\begin{enumerate}
    \item[a)] Calculer $P(\text{verte})$.
    \item[b)] Avec remise, $P(2 \text{ vertes})$.
\end{enumerate}

Réponses de l’élève :
\begin{enumerate}
    \item[a)] $P(\text{verte}) = 0.5$. \\
    \textcolor{red}{Correct, note : 2/2}
    
    \item[b)] $0.5 \times 0.5 = 0.25$. \\
    \textcolor{red}{Correct, note : 2/2}
\end{enumerate}

\textbf{Note totale : 14/20}

\end{document}