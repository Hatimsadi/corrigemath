\documentclass{article}
\usepackage[utf8]{inputenc}
\usepackage[T1]{fontenc}
\usepackage{amsmath}
\usepackage{amssymb}

\usepackage[utf8]{inputenc}
\usepackage[T1]{fontenc}
\usepackage[utf8]{inputenc}
\usepackage[T1]{fontenc}
\begin{document}

\section*{Examen de Mathématiques – Niveau 1ère (Évaluation 2)}

\subsection*{Exercice 1 : Fonctions}
Soit \( g(x) = 2x^2 - 4x + 1 \). \\
Questions : 
\begin{enumerate}
    \item[a)] Calculer \( g'(x) \).
    \item[b)] Donner les intervalles de croissance et décroissance.
    \item[c)] Résoudre \( g(x) = 0 \).
\end{enumerate}
Réponses de l’élève :
\begin{enumerate}
    \item[a)] \( g'(x) = 4x - 4 \). \\
    \textbf{Correction :} Bonne réponse. \( g'(x) = 4x - 4 \). \\
    \textbf{Note :} 2/2
    \item[b)] \( g \) est décroissante sur \((- \infty, 1]\) et croissante sur \([1, +\infty[\). \\
    \textbf{Correction :} Justification correcte mais il manque une vérification que \( g'(x) \) change de signe à \( x = 1 \). \\
    \textbf{Note :} 1.5/2
    \item[c)] \( g(x) = 0 \) approx. pour \( x = 0.5 \) et \( x = 1.5 \). \\
    \textbf{Correction :} Résolution incorrecte, il faut utiliser le discriminant. \\
    \[ g(x) = 2x^2 - 4x + 1 = 0 \]
    \[ \Delta = (-4)^2 - 4 \cdot 2 \cdot 1 = 16 - 8 = 8 \]
    \[ x = \frac{4 \pm \sqrt{8}}{4} = \frac{4 \pm 2\sqrt{2}}{4} = \frac{2 \pm \sqrt{2}}{2} \]
    \textbf{Note :} 0/2
\end{enumerate}

\subsection*{Exercice 2 : Équation}
Résoudre \( x^2 - 4 = 0 \). \\
Réponse de l’élève : 
\[
x = \pm 2.
\] \\
\textbf{Correction :} Bonne réponse. \( x = \pm 2 \). \\
\textbf{Note :} 2/2

\subsection*{Exercice 3 : Géométrie}
On considère le quadrilatère \( ABCD \) avec \( A(0,0) \), \( B(4,0) \), \( C(4,3) \) et \( D(0,3) \). \\
Questions : 
\begin{enumerate}
    \item[a)] Vérifier que \( ABCD \) est un rectangle. \\
    Réponse de l’élève : Les côtés opposés sont parallèles et égaux, c’est un rectangle. \\
    \textbf{Correction :} La justification est insuffisante. \\
    \textbf{Justification manquante :} Montrer que les diagonales sont égales ou que les angles sont droits. \\
    \textbf{Note :} 1/2
    \item[b)] Calculer son aire. \\
    Réponse de l’élève : Aire \( = 4 \times 3 = 12 \). \\
    \textbf{Correction :} Bonne réponse. Aire \( = 12 \). \\
    \textbf{Note :} 2/2
\end{enumerate}

\subsection*{Exercice 4 : Probabilités}
Dans un sac, 5 boules vertes et 5 boules jaunes. \\
Questions : 
\begin{enumerate}
    \item[a)] Calculer \( P(\text{verte}) \). \\
    Réponse de l’élève : \( P(\text{verte}) = 0.5 \). \\
    \textbf{Correction :} Bonne réponse. \( P(\text{verte}) = \frac{5}{10} = 0.5 \). \\
    \textbf{Note :} 2/2
    \item[b)] Avec remise, \( P(2 \text{ vertes}) \). \\
    Réponse de l’élève : \( 0.5 \times 0.5 = 0.25 \). \\
    \textbf{Correction :} Bonne réponse. \( P(2 \text{ vertes}) = 0.5 \times 0.5 = 0.25 \). \\
    \textbf{Note :} 2/2
\end{enumerate}

\section*{Note finale}
La note finale est : \( \frac{2 + 1.5 + 0 + 2 + 1 + 2 + 2 + 2}{10} \times 10 = 7.5/10 \).

\end{document}