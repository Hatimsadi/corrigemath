\documentclass{article}
\usepackage[utf8]{inputenc}
\usepackage[T1]{fontenc}
\usepackage{amsmath}
\usepackage{amssymb}
\usepackage{xcolor}

\usepackage[utf8]{inputenc}
\usepackage[T1]{fontenc}
\begin{document}

\section*{Exercice 1 : Analyse (5 points)}
\subsection*{1. Calculer la dérivée $f'(x)$.}
\textcolor{red}{Réponse correcte :} $f'(x) = 3x^2 - 6x$ \\
Note : 2,5/2,5

\subsection*{2. Déterminer les coordonnées du point d'intersection avec l'axe des ordonnées.}
\textcolor{red}{Réponse correcte :} $(0; 2)$ \\
Note : 2,5/2,5

\section*{Exercice 2 : Probabilités (7 points)}
\subsection*{1. Calculer la probabilité d'obtenir deux boules de la même couleur.}
\textcolor{red}{Réponse incorrecte. Correction : 
\[
P = \left(\frac{5}{8}\right)^2 + \left(\frac{3}{8}\right)^2 = \frac{25}{64} + \frac{9}{64} = \frac{34}{64}
\]}
Note : 0/4

\subsection*{2. Quelle est la probabilité d'obtenir au moins une boule rouge ?}
\textcolor{red}{Réponse correcte :} $\frac{55}{64}$ \\
Note : 3/3

\section*{Exercice 3 : Géométrie (8 points)}
\subsection*{1. Calculer les coordonnées du vecteur $-\vec{AB}$.}
\textcolor{red}{Réponse incorrecte. Correction : 
\[
-\vec{AB} = (-(4-1); -(5-2)) = (-3; -3)
\]}
Note : 0/3

\subsection*{2. Montrer que $ABC$ est un triangle rectangle.}
\textcolor{red}{Réponse correcte. Les calculs sont justes.} \\
Note : 5/5

\section*{Notes finales}
\begin{itemize}
\item Exercice 1 : 5/5
\item Exercice 2 : 3/7
\item Exercice 3 : 5/8
\item Total : \boxed{13/20}
\end{itemize}

\section*{Recommandations}
\begin{itemize}
\item \textcolor{red}{Revoyez le calcul des vecteurs et les opérations de base en géométrie.}
\item \textcolor{red}{Attention aux tirages avec remise en probabilités : les combinaisons ne sont pas utilisées ici.}
\item \textcolor{red}{Justifiez toujours vos réponses avec des calculs appropriés.}
\end{itemize}

\end{document}