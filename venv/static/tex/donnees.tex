\documentclass{article}
\usepackage[utf8]{inputenc}
\usepackage[T1]{fontenc}
\usepackage{amsmath}
\usepackage{amssymb}

\usepackage[utf8]{inputenc}
\usepackage[T1]{fontenc}
\begin{document}

\section*{Examen de Mathématiques – Niveau 1ère (Évaluation 4)}

\subsection*{Exercice 1 : Fonction et Dérivée}
Soit $f(x) = x^3 - 2x^2 - x + 2$.\\
Questions :\\
a) Calculer $f'(x)$.\\
b) Étudier les variations de $f$.\\
c) Déterminer $f(x) = 0$.\\

Réponses de l’élève :\\
a) $f'(x) = 3x^2 - 4x - 1$.\\
b) $f$ décroît puis croît (approximation).\\
c) $f(x) = (x - 1)(x^2 - x - 2) = 0$ donc $x = 1$ et $x = -1$ ou $x = 2$ (approximation).\\

\subsection*{Exercice 2 : Équation du Second Degré}
Résoudre $x^2 - 2x - 3 = 0$.\\
Réponse de l’élève :\\
$x = -1$ ou $x = 3$.\\

\subsection*{Exercice 3 : Géométrie Analytique}
On considère le triangle $ABC$ avec $A(1, 2)$, $B(5, 2)$ et $C(3, 5)$.\\
Questions :\\
a) Vérifier que le triangle est isocèle.\\
b) Calculer son aire.\\

Réponses de l’élève :\\
a) $AB = 4$ et $AC \approx 3.61$, $BC \approx 3.61$ donc isocèle.\\
b) Aire $\approx 6$.\\

\subsection*{Exercice 4 : Probabilités}
Dans un sac, 8 boules blanches et 2 boules noires.\\
Questions :\\
a) Calculer $P(\text{noire})$.\\
b) Avec remise, $P(2 \text{ noires})$.\\

Réponses de l’élève :\\
a) $P(\text{noire}) = \frac{2}{10} = 0.2$.\\
b) $0.2 \times 0.2 = 0.04$.

\end{document}