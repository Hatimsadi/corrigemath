\documentclass{article}
\usepackage[utf8]{inputenc}
\usepackage[T1]{fontenc}
\usepackage{amsmath}
\usepackage{amssymb}

\usepackage[utf8]{inputenc}
\usepackage[T1]{fontenc}
\begin{document}

\title{Examen de Mathématiques – Niveau 1ère (Évaluation 2)}
\maketitle

\section*{Exercice 1 : Fonctions}
Soit $g(x) = 2x^2 - 4x + 1.$

Questions :
\begin{enumerate}
    \item Calculer $g'(x)$.
    \item Donner les intervalles de croissance et décroissance.
    \item Résoudre $g(x) = 0$.
\end{enumerate}

Réponses de l’élève :
\begin{itemize}
    \item $g'(x) = 4x - 4.$
    \item $g$ est décroissante sur $]-\infty, 1]$ et croissante sur $[1, +\infty[$.
    \item $g(x) = 0$ approx. pour $x = 0.5$ et $x = 1.5$.
\end{itemize}

\section*{Exercice 2 : Équation}
Résoudre $x^2 - 4 = 0.$

Réponse de l’élève :
$x = \pm 2.$

\section*{Exercice 3 : Géométrie}
On considère le quadrilatère $ABCD$ avec $A(0,0)$, $B(4,0)$, $C(4,3)$ et $D(0,3)$.

Questions :
\begin{enumerate}
    \item Vérifier que $ABCD$ est un rectangle.
    \item Calculer son aire.
\end{enumerate}

Réponses de l’élève :
\begin{itemize}
    \item Les côtés opposés sont parallèles et égaux, c’est un rectangle.
    \item Aire = $4 \times 3 = 12.$
\end{itemize}

\section*{Exercice 4 : Probabilités}
Dans un sac, 5 boules vertes et 5 boules jaunes.

Questions :
\begin{enumerate}
    \item Calculer $P(\text{verte})$.
    \item Avec remise, $P(\text{2 vertes})$.
\end{enumerate}

Réponses de l’élève :
\begin{itemize}
    \item $P(\text{verte}) = 0.5.$
    \item $0.5 \times 0.5 = 0.25.$
\end{itemize}

\end{document}