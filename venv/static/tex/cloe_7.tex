\documentclass{article}
\usepackage[utf8]{inputenc}
\usepackage[T1]{fontenc}
\usepackage{amsmath}
\usepackage{amssymb}

\usepackage[utf8]{inputenc}
\usepackage[T1]{fontenc}
\begin{document}

\title{Examen de Mathématiques – Terminale}
\maketitle

\section*{Exercice 1 : Fonctions et Dérivées}
Soit $f(x) = \ln(x) - x$.\\
Questions :\\
a) Calculer $f'(x)$.\\
b) Étudier les variations de $f$ sur $(0,+\infty)$.\\
c) Trouver l’équation de la tangente à $f$ en $x = 1$.\\
Réponses de l’élève :\\
a) $f'(x) = \frac{1}{x} - 1$.\\
b) $f$ est croissante sur $(0,1)$ et décroissante sur $(1,+\infty)$ (approximation).\\
c) Pour $x = 1$, $f(1) = \ln(1) - 1 = -1$ et $f'(1) = 0$ ; donc la tangente est $y = -1$.

\section*{Exercice 2 : Limites et Continuité}
Soit $g(x) = \frac{e^x - 1}{x}$.\\
Questions :\\
a) Calculer $\lim_{x \to 0} g(x)$.\\
b) Montrer que $g$ peut être rendue continue en $x = 0$ en posant $g(0) = 1$.\\
Réponses de l’élève :\\
a) $\lim_{x \to 0} g(x) = 1$.\\
b) En posant $g(0) = 1$, la fonction devient continue en $0$.

\section*{Exercice 3 : Nombres Complexes}
Soit $z = 1 + i$ et $w = 2 - i$.\\
Questions :\\
a) Calculer $z + w$ et $z \times w$.\\
b) Donner la forme trigonométrique de $z$.\\
Réponses de l’élève :\\
a) $z + w = 3 + 0i$ et $z \times w = 3 + i$ (approximation).\\
b) $z = \sqrt{2} \left( \cos\frac{\pi}{4} + i\sin\frac{\pi}{4} \right)$.

\section*{Exercice 4 : Intégrales}
Calculer l’intégrale $I = \int_0^1 (2x + 1) \, dx$.\\
Réponse de l’élève :\\
$I = \left[ x^2 + x \right]_0^1 = 1 + 1 = 2$.

\end{document}